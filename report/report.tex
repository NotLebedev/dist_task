\documentclass[a4paper,12pt,titlepage,finall]{article}

\usepackage[T1,T2A]{fontenc}     % форматы шрифтов
\usepackage[utf8]{inputenc}      % кодировка символов, используемая в данном файле
\usepackage[english, russian]{babel}      % пакет русификации
\usepackage{tikz}                % для создания иллюстраций
\usepackage{pgfplots}            % для вывода графиков функций
\usepackage{geometry}		     % для настройки размера полей
\usepackage{indentfirst}         % для отступа в первом абзаце секции
\usepackage{amsmath,amsthm,amssymb}
\usepackage{mathtext}
\usepackage{graphicx}

\usepackage{caption}
\usepackage{subcaption}
\usepackage{hyperref}
\graphicspath{ {./img} }

%Настройка листингов
\usepackage{xcolor}
\usepackage{listings}
\lstset{
  extendedchars=\true
  basicstyle=\ttfamily,
  columns=fullflexible,
  frame=single,
  breaklines=true,
  postbreak=\mbox{\textcolor{red}{$\hookrightarrow$}\space},
}

\definecolor{mGreen}{rgb}{0,0.6,0}
\definecolor{mGray}{rgb}{0.5,0.5,0.5}
\definecolor{mPurple}{rgb}{0.58,0,0.82}
\definecolor{backgroundColour}{rgb}{0.95,0.95,0.95}

\lstdefinestyle{CStyle}{
    backgroundcolor=\color{backgroundColour},   
    keywordstyle=\color{mGreen},
    numberstyle=\tiny\color{mGray},
    breakatwhitespace=false,         
    breaklines=true,                 
    captionpos=b,                    
    keepspaces=true,                 
    numbers=none,                    
    numbersep=5pt,                  
    showspaces=false,                
    showstringspaces=false,
    showtabs=false,                  
    tabsize=2,
    language=C,
    basicstyle=\footnotesize\ttfamily ,
    extendedchars=\true ,
}

% выбираем размер листа А4, все поля ставим по 3см
\geometry{a4paper,left=30mm,top=30mm,bottom=30mm,right=30mm}

\setcounter{secnumdepth}{0}      % отключаем нумерацию секций
\setcounter{tocdepth}{2}

\usepgfplotslibrary{fillbetween} % для изображения областей на графиках

\begin{document}
\begin{titlepage}
    \begin{center}
	{\small \sc Московский государственный университет \\имени М.~В.~Ломоносова\\
	Факультет вычислительной математики и кибернетики\\}
	\hrulefill
	\vfill
	{\large \bf Компьютерный практикум по учебному курсу}\\
	~\\
	{\Large \bf <<Распределённые системы>>}\\ 
	~\\
	~\\
	~\\
	{\bf Разработка алгоритмов надёжного хранения и обработки данных}\\
	~\\
	{\large \bf ОТЧЕТ}\\
	{\bf о выполненном задании}\\
	{студента 427 учебной группы факультета ВМК МГУ}\\
	{Галустова Артемия Львовича}
    \end{center}
    
    \begin{center}
	\vfill
	{\small гор. Москва\\2022 год}
    \end{center}
\end{titlepage}

\tableofcontents
\newpage
\section{Протокол голосования}
\subsection{Постановка задачи}
Рассматривается система состоящая из клиента и $N$ файловых серверов на которых хранятся размноженные копии файлов. Требуется реализовать протокол голосования при размножении файлов. Также требуется дать оценку времени выполнения данного алгоритма.
\subsection{Описание алгоритма}
Сущность метода голосования при размножении заключается в выполнении запросе чтения и записи у многих серверов.
\par
Для $N$ серверов выбираются два числа $N_r, N_w$ - кворумы на чтение и запись, соответственно, такие что $N_r + N_w > N$. При выполнении операции чтения клиент должен обратиться к $N_r$ серверам. Каждый возвращает содержимое файла и версию. Клиент выбирает файл с максимальной версией. При выполнении операции записи требуется установить согласие между серверами относительно следующей версии файла, после чего запросить запись у всех серверов с новой версией и считать успешной, если как минимум $N_w$ серверов выполнили её.
\subsection{Оценка времени выполнения}
Требуется дать оценку времени выполнения одним процессом 2-х операций записи и 10 операций чтения $L=300$ байтов информации с файлом, расположенным на остальных $N=10$ ЭВМ сети с шинной организацией (без аппаратных возможностей широковещания). Время старта и время передачи одного байта принимаются равными $T_s=100$ и $T_b=1$ соответственно. \par
Дополнительно предположим, что версия файла и имя (индекс) файла на сервере занимают по $R=8$ байт, а тип запроса не требуется отдельно кодировать. Для чтения файла необходимо запросить у $N_r$ серверов данные. Каждому серверу будет отправлен запрос с именем файла и в ответ получен файл и его версия. Итого для каждого запроса на чтение $T_r = N_r \cdot ((T_s + T_b R) + (T_s + T_b (R + L)))$. Для чтения сперва необходимо установить новую версию файла, запросив старую у $N_r$ серверов, передав им имя файла и получив версию в ответ. После чего передать \emph{всем} серверам новый файл, его имя и новую версию и получить ответ от $N_w$ серверов (а остальные проигнорировать). Итого для каждого запроса на запись $T_w = N_r \cdot ((T_s + T_b R) + (T_s + T_b R)) + N \cdot (T_s + T_b(R + L)) + N_w T_s$. Всего запросы будут выполняться $T = 2T_w + 10T_r$. Наложим условие $N_r + N_w > N$ и произведём оптимизацию по $N_r$ и $N_w$ в $\mathbb{N}$. Как и ожидается минимум достигается при $N_r = 1$, $N_w = 10$. При этом время работы составит $T=15752$.
\subsection{Реализация}
Алгоритм был реализован на языке C++ с использованием библиотеки передачи сообщений MPI. Код реализации доступен в приложенном архиве, по адресу \url{https://github.com/NotLebedev/dist_task/tree/master/voting-protocol}, а также частично в <<\nameref{votesource}>>. Была реализована модельная система из одного клиента и нескольких серверов, обменивающихся сообщениями через систему MPI (см. файлы Client.h и Server.h).\par
Для запуска необходимо собрать проект при помощи утилиты  \texttt{CMake} командой  \texttt{cmake - -build ./cmake-build-debug - -target integrals\_ulfm}. После чего полученный файл \texttt{./cmake-build-debug/integrals\_ulfm} запускается с единственной опцией - именем конфигурационного файла (вместе с исходным кодом поставляются два файла - \texttt{faulty\_reads.toml}, \texttt{faulty\_writes.toml}, демонстрирующие обработку неуспешного чтения и записи). При помощи переменной среды \texttt{READ\_QUORUM} можно  выставить кворум на чтение (кворум на запись автоматически выбирается как $N - N_r + 1$. После запуска программа выведет отчёт о выполнении операций заданных в файле конфигурации и произошедших ошибках.
\subsection{Результаты}
Ниже приведён вывод для конфигураций \texttt{faulty\_reads.toml} и \texttt{faulty\_writes.toml} при \texttt{READ\_QUORUM} равном 3 на 8 процессах (1 клиент, 7 серверов)
\subsubsection{faulty\_reads}
\begin{lstlisting}
Read quorum is: 3
Write quorum is: 5
Started client. Reading configuration.
Files distributed before start:
	blindsight = "It didn't start out here. Not with the scramblers or Rorschach, not with Big Ben or Theseus or the vampires."
	dune = "A beginning is the time for taking the most delicate care that the balances are correct."
	farehnheit451 = "It was a pleasure to burn. It was a special pleasure to see things eaten, to see things blackened and changed."
Operations to be performed:
	Read file blindsight
	Disable server 1
	Read file blindsight
	Disable server 2
	Read file blindsight
	Disable server 3
	Read file blindsight
	Disable server 4
	Read file blindsight
	Disable server 5
	Read file blindsight

Next command:
	Read file blindsight
Read successfull. Got versions 1 1 1  Latest versions had contents:
 "It didn't start out here. Not with the scramblers or Rorschach, not with Big Ben or Theseus or the vampires."


Next command:
	Disable server 1


Next command:
	Read file blindsight
Read successfull. Got versions 1 1 1  Latest versions had contents:
 "It didn't start out here. Not with the scramblers or Rorschach, not with Big Ben or Theseus or the vampires."


Next command:
	Disable server 2


Next command:
	Read file blindsight
Read successfull. Got versions 1 1 1  Latest versions had contents:
 "It didn't start out here. Not with the scramblers or Rorschach, not with Big Ben or Theseus or the vampires."


Next command:
	Disable server 3


Next command:
	Read file blindsight
Read successfull. Got versions 1 1 1  Latest versions had contents:
 "It didn't start out here. Not with the scramblers or Rorschach, not with Big Ben or Theseus or the vampires."


Next command:
	Disable server 4


Next command:
	Read file blindsight
Read successfull. Got versions 1 1 1  Latest versions had contents:
 "It didn't start out here. Not with the scramblers or Rorschach, not with Big Ben or Theseus or the vampires."


Next command:
	Disable server 5


Next command:
	Read file blindsight
Failed to read. Quorum not met



Process finished with exit code 0
\end{lstlisting}
\subsubsection{faulty\_writes}
\begin{lstlisting}
Read quorum is: 3
Write quorum is: 5
Started client. Reading configuration.
Files distributed before start:
	blindsight = "It didn't start out here. Not with the scramblers or Rorschach, not with Big Ben or Theseus or the vampires."
	dune = "A beginning is the time for taking the most delicate care that the balances are correct."
	farehnheit451 = "It was a pleasure to burn. It was a special pleasure to see things eaten, to see things blackened and changed."
Operations to be performed:
	Read file dune
	Write contents "This every sister of the Bene Gesserit knows." to file dune
	Read file dune
	Read file blindsight
	Disable server 1
	Disable server 2
	Disable server 3
	Write contents "To begin your study of the life of Muad‘Dib, then, take care that you first place him in his time: born in the 57th year of the Padishah Emperor, Shaddam IV." to file dune
	Enable server 1
	Enable server 2
	Enable server 3
	Read file dune
	Disable server 1
	Disable server 3
	Write contents "To begin your study of the life of Muad‘Dib, then, take care that you first place him in his time: born in the 57th year of the Padishah Emperor, Shaddam IV." to file dune
	Enable server 1
	Enable server 3
	Read file dune

Next command:
	Read file dune
Read successfull. Got versions 1 1 1  Latest versions had contents:
 "A beginning is the time for taking the most delicate care that the balances are correct."


Next command:
	Write contents "This every sister of the Bene Gesserit knows." to file dune
Successful write. Updated files to version 2


Next command:
	Read file dune
Read successfull. Got versions 2 2 2  Latest versions had contents:
 "This every sister of the Bene Gesserit knows."


Next command:
	Read file blindsight
Read successfull. Got versions 1 1 1  Latest versions had contents:
 "It didn't start out here. Not with the scramblers or Rorschach, not with Big Ben or Theseus or the vampires."


Next command:
	Disable server 1


Next command:
	Disable server 2


Next command:
	Disable server 3


Next command:
	Write contents "To begin your study of the life of Muad‘Dib, then, take care that you first place him in his time: born in the 57th year of the Padishah Emperor, Shaddam IV." to file dune
Failed to write. Quorum not met


Next command:
	Enable server 1


Next command:
	Enable server 2


Next command:
	Enable server 3


Next command:
	Read file dune
Read successfull. Got versions 2 2 2  Latest versions had contents:
 "This every sister of the Bene Gesserit knows."


Next command:
	Disable server 1


Next command:
	Disable server 3


Next command:
	Write contents "To begin your study of the life of Muad‘Dib, then, take care that you first place him in his time: born in the 57th year of the Padishah Emperor, Shaddam IV." to file dune
Successful write. Updated files to version 3


Next command:
	Enable server 1


Next command:
	Enable server 3


Next command:
	Read file dune
Read successfull. Got versions 2 3 2  Latest versions had contents:
 "To begin your study of the life of Muad‘Dib, then, take care that you first place him in his time: born in the 57th year of the Padishah Emperor, Shaddam IV."



Process finished with exit code 0
\end{lstlisting}
\subsection{Выводы}
Удалось реализовать протокол голосования при работе с размноженными файлами и продемонстрировать его работу на модельной системе. Протокол корректно обрабатывает ситуации отказов файловых серверов, как в случае достаточного так и в случае недостаточного кворума.

\newpage
\section{Отказоустойчивая MPI программа}
\subsection{Постановка задачи}
Рассматривается программа, рассчитывающая определённый интеграл реализованная в рамках курса "Суперкомпьютеры и параллельная обработка данных" (см. \url{https://github.com/NotLebedev/integralsOpenMP/tree/mpi-impl}). Требуется добавить контрольные точки для продолжения работы программы в случае сбоя. Реализовать один из 3-х сценариев работы после сбоя: a) продолжить работу программы только на “исправных” процессах; б) вместо процессов, вышедших из строя, создать новые MPI-процессы, которые необходимо использовать для продолжения расчетов; в) при запуске программы на счет сразу запустить некоторое дополнительное количество MPI-процессов, которые использовать в случае сбоя.

\subsection{Описание алгоритма}
Детали реализации оригинального алгоритма доступны в старом отчёте по ссылке указанной выше. Для реализации отказоустойчивой версии важен порядок работы алгоритма: мастер-процесс (номер 0) разделяет интервал на котором осуществляется интегрирование на $N$ - число потоков, частей. После чего каждому потоку отправляется информация о его интервале (задание), производится вычисление заданий и результат редуцируется. Для реализации отказоустойчивости была выбран сценарий продолжения работы на "исправных процессах". Мастер процесс создаёт начальную контрольную точку: запоминает все задания, текущую сумму (в начале она 0); и запускает вычисление. Если после завершения вычислений был зафиксирован отказ некоторых процессов, то из контрольной точки восстанавливаются их задания, создаётся новая контрольная точка с оставшимися заданиями, промежуточной суммой (от успешно завершившихся процессов); и оставшиеся задания рассылаются оставшимся процессам. Это повторяется пока все задания не будут выполнены.

\subsubsection{Реализация}
Код реализации доступен в приложенном архиве, по адресу \url{https://github.com/NotLebedev/dist_task/tree/master/integrals-ulfm}, а также частично в <<\nameref{ulfmsource}>>.\par
Для запуска, как и для первого проекта необходимо собрать проект при помощи утилиты  \texttt{CMake}. Команда  \texttt{cmake - -build ./cmake-build-debug - -target voting\_protocol}. После чего полученный файл \texttt{./cmake-build-debug/voting\_protocol} запускается с двумя опциями - количеством шагов алгоритма интегрирования (позволяет регулировать время выполнения) и количеством процессов, которые откажут во время выполнения.

\subsection{Результаты}
Ниже приведены результаты запуска без отказов, с одним отказом и отказом всех процессов кроме трёх. Тестирование выполнялось на 12 процессах
\subsubsection{Без отказов}
\begin{lstlisting}
Unfinished jobs {}
All done!
Runtime: 4.95214 seconds. Result: 3.3504

Process finished with exit code 0
\end{lstlisting}

\subsubsection{Один отказ}
\begin{lstlisting}
Rank 10 / 12: Notified of error MPI_ERR_PROC_FAILED: Process Failure. 1 found dead: { 11 }
Rank 8 / 12: Notified of error MPI_ERR_PROC_FAILED: Process Failure. 1 found dead: { 11 }
Rank 3 / 12: Notified of error MPI_ERR_PROC_FAILED: Process Failure. 1 found dead: { 11 }
Rank 6 / 12: Notified of error MPI_ERR_PROC_FAILED: Process Failure. 1 found dead: { 11 }
Rank 2 / 12: Notified of error MPI_ERR_PROC_FAILED: Process Failure. 1 found dead: { 11 }
Rank 4 / 12: Notified of error MPI_ERR_PROC_FAILED: Process Failure. 1 found dead: { 11 }
Rank 0 / 12: Notified of error MPI_ERR_PROC_FAILED: Process Failure. 1 found dead: { 11 }
Rank 5 / 12: Notified of error MPI_ERR_PROC_FAILED: Process Failure. 1 found dead: { 11 }
Rank 7 / 12: Notified of error MPI_ERR_PROC_FAILED: Process Failure. 1 found dead: { 11 }
Rank 9 / 12: Notified of error MPI_ERR_PROC_FAILED: Process Failure. 1 found dead: { 11 }
Rank 1 / 12: Notified of error MPI_ERR_PROC_FAILED: Process Failure. 1 found dead: { 11 }
Unfinished jobs {11, }
Unfinished jobs {}
All done!
Runtime: 8.26728 seconds. Result: 3.3504

Process finished with exit code 0
\end{lstlisting}

\subsubsection{Отказ всех кроме трёх}
\begin{lstlisting}
Rank 0 / 12: Notified of error MPI_ERR_PROC_FAILED: Process Failure. 9 found dead: { 3 4 5 6 7 8 9 10 11 }
Rank 1 / 12: Notified of error MPI_ERR_PROC_FAILED: Process Failure. 9 found dead: { 3 4 5 6 7 8 9 10 11 }
Rank 2 / 12: Notified of error MPI_ERR_PROC_FAILED: Process Failure. 9 found dead: { 3 4 5 6 7 8 9 10 11 }
Unfinished jobs {3, 4, 5, 6, 7, 8, 9, 10, 11, }
Unfinished jobs {3, 4, 5, 6, 7, 8, }
Unfinished jobs {3, 4, 5, }
Unfinished jobs {}
All done!
Runtime: 14.8369 seconds. Result: 3.3504

Process finished with exit code 0
\end{lstlisting}
\subsection{Выводы}
Удалось модифицировать существующую программу, чтобы добиться отказоустойчивого поведения и не изменить результаты вычисления. Программа может справиться с отказом единичных процессов, а также отказом большей части процессов.
\newpage
\section{Исходный код для протокола голосования}\label{votesource}
\subsection{Файл Client.h}
\lstinputlisting[style = CStyle]{../voting-protocol/Client.h}
\newpage
\subsection{Файл Client.cpp}
\lstinputlisting[style = CStyle]{../voting-protocol/Client.cpp}
\newpage
\subsection{Файл Server.h}
\lstinputlisting[style = CStyle]{../voting-protocol/Server.h}
\newpage
\subsection{Файл Server.cpp}
\lstinputlisting[style = CStyle]{../voting-protocol/Server.cpp}
\newpage
\section{Исходный код для отказоустойчивой MPI программы}\label{ulfmsource}
\subsection{Файл main.cpp}
\lstinputlisting[style = CStyle]{../integrals-ulfm/main.cpp}

\end{document}